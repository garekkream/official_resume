%!TEX TS-program = xelatex
\documentclass[]{friggeri-cv}

\begin{document}
\header{Krzysztof }{Garczyński}
       {embedded software developer}


% In the aside, each new line forces a line break
\begin{aside}
  \section{about:}
	phone: 888-820-151  	
    ~
   	\href{http://pl.linkedin.com/pub/krzysztof-garczynski/10/b21/782/}{linkedin profile}
   	~
   	\href{http://github.com/garekkream}{github}
	~
	\href{mailto:krzysztof.garczynski@gmail.com}{krzysztof.garczynski}
	\href{mailto:krzysztof.garczynski@gmail.com}{@gmail.com}
	
  \section{languages:}
    \textbf{english}
    (fluent)
    \textbf{russian}
    (basics)
    
  \section{programming:}
    C, C++, python, Rust
	VHDL
	\textbf{embedded linux kernel}
	\textbf{u-boot}
	android
	
  \section{skills:}
   	Architecture:
   	\textbf{ARM}, AVR, DSP}
   	~
	Protocols:
	\textbf{DECT}, WDCT}}
	I2C, SPI
	~
	Others:
  	git, ClearCase, svn
  	buildroot, \textbf{Yocto}
 	\textbf{electronic design}
 	lauterbach
 	\textbf{SCRUM}
  	~
  	
  \section{at work:}
  	faithful
  	teamworker
  	\textbf{"can-do" attitude}
  	
  \section{after hours:}
    squash player
    rock listener
    basketball fanc	
\end{aside}

\section{career goal}

\textbf{embedded linux and drivers developer}, embedded devices\\

\section{experience}

\begin{entrylist}

  \entry
    {10.2015 - ...}
    {\large{embedded linux kernel \& u-boot developer}}
    { \\ \textbf{\medium{Nokia Networks - Radio Module}}}
    {\emph{
    	\begin{itemize}
    	    \item[•]    \textbf{OS/Boot Technical Team Leader},
    	    \item[•]    Linux Trainings (over 100 people),
    	    \item[•]    Porting Linux and U-Boot to new SoC,
    	    \item[•]    Drivers development (Bootloader and OS),
    	    \item[•]    Specification and manuals for new components,
    	    \item[•]    Team interface to the Product Owners and Scrum Masters
		\end{itemize}	    
	}}


  \entry
    {07.2013 \\- 10.2015}
    {\large{embedded linux kernel \& u-boot developer}}
    { \\ \textbf{\medium{Nokia Networks - System Module}}}
    {\emph{
    	\begin{itemize}
    	    \item[•]    \textbf{Bootloaders team leader},
    	    \item[•]    U-boot porting and features development,
    		\item[•]	Linux Drivers implementation for new hardware,
    		\item[•]    Coaching new members
		\end{itemize}	    
	}}

  \entry
    {11.2010 \\- 07.2013}
    {\large{embedded software developer}}
    { \\ \textbf{\medium{Gigaset Communications Sp. z.o.o.}}}
    {\emph{
    	\begin{itemize}
    		\item[•]	Embedded systems developer (DECT devices),
		    \item[•]	Features design, implementation and maintenace,
			\item[•]	Preparing specifications, manuals, workpackages for test department,
			\item[•]	Maintain DSP areas (API, CPU vendors contact) and DECT embedded OS
		\end{itemize}	    
	}}
		

  \entry
    {02.2012 \\ - 01.2013}
    {\large{embedded software developer}}
    {\\\textbf{\medium{Wroclaw University Of Technology, Division of Micro- and Nanostructures Metrology}}}
    {\emph{
	\begin{itemize}
	    \item[•]	Architecture of new measurment embedded platforms for AFM
	    \item[•]    Lecturer at "ArmAcademy" workshops for students,
	    \item[•]	Students projects supervision and inspection (code reviews)
	\end{itemize}	    
    
}}
		
  \entry
    {01.2010 \\- 09.2010}
    {\large{technical engineer}}
    {\\\textbf{\medium{Wroclaw University Of Technology, Division of Micro- and Nanostructures Metrology}}}
    {\emph{
    	\begin{itemize}
        	\item[•] Embedded software developer of microprocessors (ARM, DSP), microcontrollers (Atmel) and FPGA, CPLD,
    		\item[•] Design and assembly of digital electronics circuts and devices,
			\item[•] Measurements using atomic force and scanning tunneling microscopes
    	\end{itemize}
   	}}

\end{entrylist}

\pagebreak

\section{education}

\begin{entrylist}
  \entry
    {2005–2010}
    {Faculty of Microsystem Electronics and Photonics}
    {\\ Wrocław University of Technology}
    {M.Sc. in Electronics\\
    specialization in Microsystem Electronics}
\end{entrylist}

\section{trainings}
\begin{entrylist}
	\entry
		{03.2015}
		{Kernel internals and debugging}
		{\href{http://training.linuxfoundation.org/linux-courses/development-training/linux-kernel-internals-and-debugging}{description}}
		{Linux Foundation}
	\entry
		{04.2012}
		{SCRUM for beginners}
		{\href{http://brasswillow.pl/warsztaty-i-szkolenia/wstep-do-scrum/}{PL description}}
		{Brass Willow}
	\entry
		{04.2012}
		{Linux kernel and drivers development}
		{\href{http://www.alx.pl/en/courses/linux-driver-development/}{description}}
		{ALX 2012}
	\entry
		{08.2012}
		{Android Applications}
		{\href{http://www.alx.pl/en/courses/android-101-201/}{description}}
		{ALX 2012}		
\end{entrylist}

\section{publications}
  \entry
    {\textbf{Garczyński K.}, Zielony M., Zawierucha P., Kopiec D., Gotszalk T., \\}
    {\textit{ATmega128 and CPLD based universal driver for Scanning Tunneling Microscopy} \\ }
    {Polish Electronics Conference, Darlówko Wschodnie 2010, published: Elektronika, 9/2010}


  \entry
    {Kopiec D., Jóźwiak G., Zawierucha P., \textbf{Garczyński K.}, Grabiec P., Gotszalk T., \\}
	{\textit{Calibrating spring constant known mass method} \\ }
    {Nano and Micromechanics Conference, Krasiczyn 2010}
    
    \\
\section{activities}
	SPENT member - Stowarzyszenie Polskich Entuzjastów Nanotechnologii ("Polish Nanotechnology Enthusiasts Association")

\end{document}
